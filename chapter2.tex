\pagebreak
\section*{Chapter 2}
\begin{enumerate}
    % Question 1.
    \item \textcolor{question}{
        Check whether any of the following processes is a Markov process. Justify your answer.
    }
    \begin{enumerate}
        % Part (a)
        \item \textcolor{question}{
            $(Y_n)_{n\in \mathbb N}$ given by 
            $Y_n = \sum_{i=0}^n X_i$, where $X_i$ are \iid\ random variables.
        }

        Let $\mathbb F$ be the natural filtration of $Y$, so $\mathcal F_t = \sigma (Y_s:s\leq t)$.

        \textbf{Showing that $F_t=\sigma(X_s:s\leq t)$}\\
        TODO:

        \textbf{Showing that $\sum_{i=s+1}^t X_i$ is independent to $F_s$}\\
        TODO:

        \textbf{Proving Markov Property}\\
        Let $f:\mathbb{R}\to\mathbb{R}$ be any function. Then the conditional expectation
        \begin{align*}
            \mathbb E\left[f\left(Y_t\right)\middle|\mathcal F_s\right] 
            &= \mathbb E\left[f\left(Y_s + \sum_{i=s+1}^t X_i \right)\middle|\mathcal F_s\right]\\
            &= \mathbb E\left[f\left(y + \sum_{i=s+1}^t X_i \right)\right]\bigg|_{y=Y_s},
        \end{align*}
        which only depends on $\omega$ through $Y_s$, and thus is $\sigma(Y_s)$-measurable.\\
        Therefore $Y$ is $\mathbb F$-Markov.

        \pagebreak

        % Part (b)
        \item \textcolor{question}{
            $B_t=\frac{1}{2}(W_{t+a}-W_t)$, where $a>0$, and $W$ is 
            a Wiener process with natural filtration $\mathbb F$.
        }

    \end{enumerate}
\end{enumerate}