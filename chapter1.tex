\section*{Chapter 1}
\begin{enumerate}
    % Question 1.
    \item \textcolor{question}{Let $(\Omega,\mathcal F,\mu)$ be a measure space.}
    \begin{enumerate}
        % Part (a)
        \item \textcolor{question}{Show that a $\sigma$-algebra is closed under countable intersection.}


        Suppose $A_i \in \mathcal F$ for a countable index set $i\in I$.
        \begin{align*}
            (\forall i\in I) (A_i^c &\in \mathcal F) &\text{(Closure under complement)}\\
            \bigcup_{i\in I} A_i^c &\in \mathcal F &\text{(Closure under countable union)}\\
            \therefore\bigcap_{i\in I} A_i = \left(\bigcup_{i\in I} A_i^c\right)^c &\in \mathcal F &\text{(Closure under complement)}.
        \end{align*}

        % Part (b)
        \item \textcolor{question}{Discuss the difference between countable additivity and countable sub-additivity of $\mu$.}

        Any measure $\mu$ must satisfy countable additivity (by definition):\\
        Given a countable collection of \emph{disjoint} sets $A_i\in \mathcal F$,
        $$\mu\left(\bigcup_{i\in I} A_i\right) = \sum_{i\in I}\mu(A_i).$$
        It can be proven that measure $\mu$ must satisfy countable sub-additivity:\\
        Given a countable collection of \emph{(not necessarily disjoint)} sets $A_i\in \mathcal F$,
        $$\mu\left(\bigcup_{i\in I} A_i\right) = \sum_{i\in I}\mu(A_i).$$

        % Part (c)
        \item \textcolor{question}{Find an example of $\mu$ and a non-empty null $\mu$-null set.}

        Let $\Omega$ be be any non-empty set, and let $\mathcal F = \{\emptyset,\Omega\}$ be the trivial $\sigma$-algebra.\\
        Define the set function $\mu:\mathcal F\to [0,\infty]$ by $\mu(\emptyset) = \mu(\Omega) = 0$.\\
        Then $\mu$ is a measure (countable additivity is trivial), and $\Omega$ is a non-empty $\mu$-null set.

        \pagebreak

        % Part (d)
        \item \textcolor{question}{Let $\nu:\mathcal F \to [0,\infty]$ be a measure.
        Let $a,b\in \mathbb R$. 
        Find conditions under which $\pi = a\mu + b\nu$ defines a measure.
        Justify your answer.}

        Firstly $\pi$ must map elements in $\mathcal F$ to non-negative reals.
        Thus $a,b$ must be such that $a\mu + b\nu\geq 0$ everywhere. 

        Secondly, $\pi$ must map the empty set to $0$.\\
        This does not impose any extra restrictions on $a,b$, as
        \begin{align*}
            \pi(\emptyset)
            &= a\mu(\emptyset) + b\mu(\emptyset)\\
            &= 0.
        \end{align*}

        Lastly, $\pi$ must satisfy countable sub-additivity.\\
        Again, this does not impose any extra restrictions, as for any $a,b\in \mathbb R$,
        \begin{align*}
            \pi\left(\bigcup_{i\in i}a_i\right)
            &= a\mu\left(\bigcup_{i\in I}A_i\right) + b\mu\left(\bigcup_{i\in I}A_i\right)\\
            &= a\sum_{i\in I}\mu(A_i) + b\sum_{i\in I}\nu(A_i)\\
            &= \sum_{i\in I} \pi (A_i).
        \end{align*}
        Thus for $\pi$ to define a measure, it is sufficient and necessary for $a\mu +b\nu\geq 0$.
    \end{enumerate}

    \pagebreak

    % Question 2.
    \item \textcolor{question}{Let $X,Y:\Omega\to\mathbb R$ be (continuous) random variables on $(\Omega,\mathcal F,\mathbb P)$.}
    \begin{enumerate}
        % Part (a)
        \item 
        \textcolor{question}{Write down the domain and co-domain of the distribution measure $\mu_X$ of $X$. 
        Then express $\mathbb P(X\in[a, b])$ in terms of $\mu_X$, the c.d.f. $F_X$ and the p.d.f $f_X$.}

        The distribution measure $\mu_X:\mathcal F\to [0,1]$ is defined by $\mu_X=\mathbb P\circ X^{-1}$.
        $$\mathbb P(X\in[a,b])=\mu_X([a,b])=F_X(b)-F_X(a) = \int_a^b f_X(x)\,\diff x.$$

        % Part (b)
        \item 
        \textcolor{question}{
            Write down $\mu_X, F_X$ and $f_X$ if
        }

        \begin{enumerate}
            \item \textcolor{question}{$X\sim \text{Uniform}([a,b])$}
            
            $$\mu_X= A\mapsto \frac{1}{b-a}\lambda(A\cap [a,b]).$$

            \item \textcolor{question}{$X\sim \mathcal N(a,\sigma^2)$}

            $$\mu_X = A\mapsto \int_A \frac{1}{\sqrt{2\pi\sigma^2}} \exp\left(-\frac{1}{2}\frac{(x-a)^2}{\sigma^2}\right)\,\lambda(\diff x).$$
        \end{enumerate}

        % Part (c)
        \item
        \textcolor{question}{
            For the random vector $(X,Y)$, write down the joint $\mu_{X,Y}$ and $f_{X,Y}$
            and the conditional $\mu_{X|Y}$ and $f_{X|Y}$ if
            $X\sim\mathcal N(0,1)$ and $Y\sim\mathcal N(a,\sigma^2)$ are jointly normal and correlated.
        }
    \end{enumerate}


\end{enumerate}