\documentclass[12pt]{article}
\usepackage[margin=2cm]{geometry}
\usepackage{tikz}
\usepackage{amsmath}
\usepackage{amssymb}
\usepackage{mdframed}
\usepackage{parskip}
\usepackage{xcolor}
\usepackage{graphicx}
\usepackage{float}
\usepackage{enumitem}
\usepackage{pdfpages}
\usepackage[hidelinks]{hyperref}

% general macros
\DeclareMathOperator{\diff}{d \!}
\DeclareMathOperator{\Loperator}{L}
\renewcommand{\L}{\Loperator}
\newcommand{\intto}{\int_{t_0}}

% GREY BOXES
\newmdenv[%
  backgroundcolor=gray!20, % Grey background color
  linewidth=1pt,           % No frame line
  linecolor=black,
  roundcorner=10pt,        % Rounded corners
  skipabove=10pt,       % Space above the box
  skipbelow=10pt,        % Space below the box
  innerleftmargin=10pt,     % Left padding
  innerrightmargin=10pt,    % Right padding
  innertopmargin=10pt,      % Top padding
  innerbottommargin=10pt    % Bottom padding
]{greybox}

\newmdenv[%
  backgroundcolor=gray!10, % Grey background color
  linewidth=0pt,           % No frame line
  roundcorner=10pt,        % Rounded corners
  skipabove=10pt,       % Space above the box
  skipbelow=10pt,        % Space below the box
  innerleftmargin=10pt,     % Left padding
  innerrightmargin=10pt,    % Right padding
  innertopmargin=10pt,      % Top padding
  innerbottommargin=10pt    % Bottom padding
]{lightbox}
\begin{document}
\section*{Chapter 1}
\begin{enumerate}
    % Question 1.
    \item Let $(\Omega,\mathcal F,\mu)$ be a measure space.
    \begin{enumerate}
        % Part (a)
        \item Show that a $\sigma$-algebra is closed under countable intersection.

        Suppose $A_i \in \mathcal F$ for a countable index set $i\in I$.
        \begin{align*}
            (\forall i\in I) (A_i^c &\in \mathcal F) &\text{(Closure under complement)}\\
            \bigcup_{i\in I} A_i^c &\in \mathcal F &\text{(Closure under countable union)}\\
            \therefore\bigcap_{i\in I} A_i = \left(\bigcup_{i\in I} A_i^c\right)^c &\in \mathcal F &\text{(Closure under complement)}.
        \end{align*}

        % Part (b)
        \item Discuss the difference between countable additivity and countable sub-additivity of $\mu$.

        Any measure $\mu$ must satisfy countable additivity (by definition):\\
        Given a countable collection of \emph{disjoint} sets $A_i\in \mathcal F$,
        $$\mu\left(\bigcup_{i\in I} A_i\right) = \sum_{i\in I}\mu(A_i).$$
        It can be proven that measure $\mu$ must satisfy countable sub-additivity:\\
        Given a countable collection of \emph{(not necessarily disjoint)} sets $A_i\in \mathcal F$,
        $$\mu\left(\bigcup_{i\in I} A_i\right) = \sum_{i\in I}\mu(A_i).$$

        % Part (c)
        \item Find an example of $\mu$ and a non-empty null $\mu$-null set.

        Let $\Omega$ be be any non-empty set, and let $\mathcal F = \{\emptyset,\Omega\}$ be the trivial $\sigma$-algebra.\\
        Define the set function $\mu:\mathcal F\to [0,\infty]$ by $\mu(\emptyset) = \mu(\Omega) = 0$.\\
        Then $\mu$ is a measure (countable additivity is trivial), and $\Omega$ is a non-empty $\mu$-null set.

    \end{enumerate}

\end{enumerate}
\end{document}